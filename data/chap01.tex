\chapter{引言}

在现今的大数据时代,随着当今智能手机以及其他的移动智能设备在全球范围爆炸式的普及,越来越多的基于地理位置信息的空间数据和基于时间及地理信息的时空数据不断地产生。而随着各式各样的微博客,各式各样的点评软件以及社交软件,越来越多的文本数据都和空间位置信息紧密的结合在了一起。在这些空间文本数据数据上展开的数据分析已经在各个领域取得了广泛的应用,他们在社会各个领域的重要性也已经变得越来越不可忽视。

举例来说,全球普遍使用的微博客软件Twitter就是一个能产生大量结合了文本与空间位置信息的数据的应用。基于Twitter的数据分析的应用更是十分广泛:比如在不久之前举世瞩目的美国大选中,我们可以通过分析Twitter,结合其中的地理位置信息和Twitter本身的内容,来估计在某个特定的州,支持希拉里和支持特朗普的人的比例为多少。又比如当社会上某个热点事件发生时,比如某期的苹果发布会,我们可以通过这些Twitter的数据,来分析这次的苹果发布会在各个地区受到的关注度是多少,从而进一步对不同的地方采取不同的销售策略和宣传策略。

当然,在这个信息化已经深深渗入了人们生活的各个角落的时代,可以用来做空间文本信息分析的应用已经不再局限于微博客这一类经典的软件。现今,各式各样的数据分析已经被广泛用作提高商业竞争力的重要手段,而针对空间文本信息的分析则可以将这一竞争力提升到一个新的高度。就拿我们现今最常使用的滴滴打车来说,我们每个滴滴打车的注册用户都会渐渐地被系统贴上一些根据我们打车特点而创立的标签。在一些车辆拥堵的高峰期,滴滴打车的调度系统将通过分析不同区域内用户的标签信息,来预计不同区域的车辆需求量以及相应的收益比,从而更好地实现车辆调度而使利益获得最大化。

除了上面提到的事实上已经算是比较传统的数据分析场景之外,最近新型的很多基于位置的应用服务(LBS,Location Based System)不断兴起。这又为空间文本数据分析提供了更多的应用。一个很典型的例子就是foursquare。 Foursquare是一款将游戏,社交元素融入到位置服务的软件。在这个软件上,用户通过实际到达某一个地点,然后在该地点签到来获得在软件中的一些虚拟头衔作为奖励。比如在同一个地点签到最多的用户,可以获得“市长”的称号。又比如,在体育馆签到最多的用户可以获得一个“健身鼠”的头衔。这些虚拟头衔很大地促进了用户在这款软件上的社交欲望。软件还提供了让用户在特定地点发表博客的功能,以及在特定的餐厅,商店留下点评的功能,并允许用户联系到过同一个地点的其他用户并与之建立朋友关系。久而久之,这就形成了一个基于实时地理位置的巨大社交网络。对与foursquare合作的商家而言,通过对这些数据的空间信息的分析,他们可以清楚的直到每天都有多少人在自己商店的周围路过,而对于这些数据文本信息的解析,他们可以找出路过的这些人中可能会对自己的商品感兴趣的潜在用户。此外,Foursquare本身也会利用这些数据来改善产品的用户体验。Foursquare的推荐系统可以通过对用户在特定地点发表的博客的分析,来找出可能有相似爱好的用户群体。当用户对一个特定的地方感兴趣的时候,系统可以通过建立一个简单的查询,查出对应的地点中有多少个和该用户相似的用户到达过,来决定系统给该用户关于该地方的推荐指数。

因为各类数据分析在长时间以来极为广泛的应用场景,在这个领域上各式各样的数据分析都已经有了很深入的研究。但随着数据集在当今时代爆炸式的增长,极大规模的数据量又给我们带来了更多的挑战。具体来说,当处理的数据规模极大的时候,找到所有的满足特定查询要求的数据点将会带来昂贵的系统开销,而如果要再继续在这些数据点之上展开分析,工作量更是不言而喻。更糟的情况下,应对多个查询而需要查找大量数据的时候,很可能出现了内存不足而需要将部分数据存到外存的情况,这时候的运行效率将会非常之低。因此在超大规模数据上执行数据分析可能会需要等待很长的时间。

其实在实际应用场景中,用户通常是不需要一个十分精确的结果的。因为实际上数据本身也带有很大的随机性和偶然性。在大多数情况下,用户会满意于获得一个近似的结果,尤其是当这个近似结果保证和真实结果偏差不大的时候。在最近的两年,开始出现了针对这样超大规模的数据实行近似查询分析的研究,其中有一篇~\cite{RS}提出了交互式分析的概念。这一概念的提出是为了满足用户在线查询分析数据并得到一个随着时间不断精确的结果的需求。在交互式分析的理念中,每个用户在进行数据分析查询的时候都有一个可控制的开关,让用户可以随时终止查询并得到结果。这样一个开关允许用户按自己的感觉和自己的理解去平衡查询等待的时间以及查询结果的质量,因而在实际应用中可以获得很好的用户体验。

\section{交互式分析在空间文本数据上的应用}

随着各式各样的社交软件以及基于位置的应用服务的兴起于普及,越来越多的适用于在空间文本数据上进行交互式分析的应用情景也不断涌现。这些软件与服务越来越多地将数据分析这一个原本只会被大型商业公司以及数据科学家触及的行为普及到了更多更广的人当中。对于普通人群而言,数据分析的交互体验就显得更加的重要了。

拿Foursquare应用举例。假设有一家零售书店的老板正在与Foursquare合作,他想利用Foursquare提供的用户数据来决定自己想要购进哪些书籍,就必须通过数据去理解在自己周围的可能会来自己书店的那块地区,人们可能关注的是什么东西。这时候他需要做的就是先列一个计划购入的书籍的候选列表,然后针对列表中的每一本书籍,找出其对应的关键词,可以是书名,人物名等等,然后指定自己所在的这块区域,指定一系列关键词,然后搜索在自己这块区域内有多少人提到了,也就是关注到了相关的内容,从而决定自己是不是要购进这本书。如果这一个过程需要书店的老板每发起一次查询就等待好一段时间,那么这一过程对于书店老板来讲就变得相当枯燥也相当难以忍受。从另一方面来说,他的分析效率实际上是非常低下的。而这样一个查询如果是一个交互式分析的查询,书店老板能在输入查询要求后马上就得回一个近似的结果,并能看到通过不断计算结果变得越来越精确的过程,那么其体验就会大大增加。更重要的是,书店老板想判断的实际上只是是否需要购进这本书,所以对于一些特别受欢迎或者特别不受欢迎的书籍,他可以在查询早期的时候就通过近似结果得出结论,从而可以大大的节约查询时间。

事实上,交互式分析除了能带来极高的用户体验之外,其带来的效率对数据分析实用性的提升作用也是巨大的。通常情况下,因为面临巨大数量的数据分析,快速地得到分析结果变得尤为关键。比如一家热点新闻评论网站,他们需要为国内热点新闻,国际热点新闻,以及各个省或者各个州的热点事件这几个不同的版块提供相应的评论。那么为了能够实时地,以最快的速度找到热点事件,热点新闻并发表评论,他们需要实时地对以上各个分类对应的各个地区的人产生的博客或者各类社交信息进行分析(通常人们会第一时间倾向于在网络上讨论自己关注的热点信息),来探测热点事件的发生。在这种情境下,系统就会需要产生大量的基于空间数据的文本分析查询,并且是基于很大数据量下的查询,一般来说,这样的实时查询是无法做到的。但实际上,这家新闻评论网站的目的仅仅只是探测热点事件的发生而已,寻求精确的查询结果是根本不必要的,因为热点事件的文本出现频率会显著的高于其他文本。这时候,在这些空间文本数据上运行交互式分析算法,然后设定一个查询准确率,交互式分析算法将会在运行到准确率满足要求的时候立刻终止并返回结果。而对于大数据而言,大部分的准确度都是可以在非常短的时间内达成的,例如说在我们最近的针对Twitter的文本分析实验中,平均检索20\%的数据就可以获得92\%以上的平均准确率。这对于一些数值分析的实验还会更高。因此在探测热点事件这样对准确率要求较低的查询中,我们可以通过交互式查询而节约大量的时间。因此,这样的算法使得牺牲小幅准确率,从而大幅度提升效率成为可能。

\section{论文贡献}

在本篇文章中,我们详细地定义和梳理了基于大规模空间数据的交互式文本分析问题。在此之上,我们展示了之前的一些关于交互式分析的一些研究以及他们在处理这个问题上展现的诸多不适应与不足之处。然后,我们探索了一种和以往类似的研究完全不同的算法框架,局部敏感哈希算法,并成功地将其应用到了这个问题中来。最后,我们提出了一种新的局部敏感哈希算法,将其融入我们的框架中使得问题得以高效地解决。

总结来说:

\begin{enumerate}[$ \bullet $]
	\item 在章节\ref{problemdefinition}中,我们规范化地定义了什么是基于大规模空间数据的交互式文本分析。然后回顾了一些现有的用来解决基于空间数据的交互式分析的算法LS树算法以及RS树算法,并将这些算法用作我们的比对算法来评估我们的效率。
	
	\item 与LS树算法以及RS树算法通过扩展R树来实现交互式分析的思路不同,鉴于文本分析的复杂性,我们采用了一种截然不同的算法框架,使用局部敏感哈希算法来解决基于空间数据的交互式分析问题。 在章节\ref{LSHIntro}以及章节\ref{LSHdiscuss}中,我们展示了一些现有的局部敏感哈希算法,并详细地讨论了之前的方法在解决本文问题上的弊端以及应用局部敏感哈希算法的可行性。
	
	\item 在章节\ref{lsh1}以及\ref{lsh1}中,我们提出并系统地构建了将局部敏感哈希算法应用到解决基于空间数据的交互式分析问题的框架。
	
	\item 在章节\ref{C3LSH}中,我们详细地讨论了我们在前一部分构建的框架在超大规模的数据下可能遇到的瓶颈:空间复杂度问题以及触碰数据时间成本问题,并设计了一种新的局部敏感哈希算法,复合型冲突计数局部敏感哈希来解决这个问题。其核心思想就是通过多层哈希函数的符合来提高整个算法使用局部敏感哈希函数的效率。
	
	\item 在章节\ref{experiment}中,我们首先详细地讨论了关于我们算法中存在的众多参数的选取问题,并建立了一套评估算法准确度和时间性能的参数模型。然后将我们算法的性能和现在最快的交互式分析算法RS树的性能进行比对,发现局部敏感哈希算法比起之前的算法在解决这个问题上有着质的飞跃。
\end{enumerate}





